%mainfile: rapport.tex

\section{Pr�sentation du sujet}


Le choix de notre projet SR05 est de reproduire un r�seau social tel que Twitter de mani�re r�parti au travers de AirPlug.\\

Twitter est un r�seau social permettant � ses membres de publier sur internet de cours messages (140 caract�res). Les utilisateurs de Twitter peuvent s'abonner � d'autres membres du r�seau social dans le but de pouvoir lire les messages r�dig�s par ces derniers.
Twitter est souvent utilis� par de grandes entreprises, des politiciens afin de pouvoir �tre lut par plusieurs millions de personnes � travers le monde, mais aussi pour avoir un retour sur leur actions. Il n'est pas rare de voir un service apr�s vente tel que celui de Dell, vous contactez si vous vous plaigniez d'un probl�mes avec un de leurs produits sur Twitter.\\

Dans le cadre de SR05, nous allons faire de PIE, un r�seau social proche de celui de Twitter avec cependant quelques diff�rences.
En effet au lieu de fonctionner avec un syst�me de serveurs centralis�s au travers d'internet, nous allons utiliser un r�seau un syst�me r�parti compos� d'un ensemble de v�hicules �quip�s d'AirPlug et de l'application PIE.
Un programme tel que PIE apporte un plus lors d'un trajet en voiture, non seulement nous pouvons suivre le flux de nos amis qui font route avec nous (convoi), ou de tout autre inconnu, ainsi nous pouvons �changer un grand nombre d'informations avec les usagers nous entourants (dans la zone de diffusion). Gr�ce au syst�me d'abonnement nous pouvons s�lectionner les flux qui nous int�resses (coll�gues, informations routi�re, famille)  tout en ignorants les autres (publicit�s, utilisateurs inint�ressants ...).