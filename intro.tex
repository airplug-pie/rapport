%mainfile: rapport.tex

\section{Présentation du sujet}

\paragraph{}
Le choix de notre projet SR05 est de reproduire un réseau social tel que Twitter de manière réparti au travers de AirPlug.

\paragraph{}
Twitter est un réseau social permettant à ses membres de publier sur internet de cours messages (140 caractères). Les utilisateurs de Twitter peuvent s'abonner à d'autres membres du réseau social dans le but de pouvoir lire les messages rédigés par ces derniers.
Twitter est souvent utilisé par de grandes entreprises, des politiciens afin de pouvoir être lut par plusieurs millions de personnes à travers le monde, mais aussi pour avoir un retour sur leur actions. Il n'est pas rare de voir un service après vente tel que celui de Dell, vous contactez si vous vous plaigniez d'un problèmes avec un de leurs produits sur Twitter.

\paragraph{}
Dans le cadre de SR05, nous allons faire de PIE, un réseau social proche de celui de Twitter avec cependant quelques différences.
En effet au lieu de fonctionner avec un système de serveurs centralisés au travers d'internet, nous allons utiliser un réseau un système réparti composé d'un ensemble de véhicules équipés d'AirPlug et de l'application PIE.
Un programme tel que PIE apporte un plus lors d'un trajet en voiture, non seulement nous pouvons suivre le flux de nos amis qui font route avec nous (convoi), ou de tout autre inconnu, ainsi nous pouvons échanger un grand nombre d'informations avec les usagers nous entourants (dans la zone de diffusion). Grâce au système d'abonnement nous pouvons sélectionner les flux qui nous intéresses (collègues, informations routière, famille)  tout en ignorants les autres (publicités, utilisateurs inintéressants\ldots).

