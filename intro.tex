%mainfile: rapport.tex

\section{Présentation du sujet}

\paragraph{}

Le choix de notre projet SR05 est de reproduire un réseau social tel que \twitter, de manière répartie, par le biais du système \airplug.

\paragraph{}

\twitter{} est un réseau social permettant à ses membres de publier sur \internet{} de courts messages, de l'ordre de 140 caractères.
Ses utilisateurs peuvent s'abonner à d'autres membres du réseau social, dans le but de pouvoir lire les messages rédigés par ces derniers. L'utilisation qui est faite de la plateforme est variée : garder le contact avec ses amis, discuter, publier une revue de presse personnelle ou encore couvrir des évènements publics en temps réel.

\twitter{} est par ailleurs utilisé par de grandes entreprises, des politiciens, dans le but de pouvoir être lus par plusieurs millions de personnes à travers le monde, mais aussi pour avoir un retour sur leurs actions.
Il n'est pas rare de voir un service après-vente tel que celui de \dell, vous contacter si vous vous plaignez d'un problème avec un de leurs produits sur \twitter.

\paragraph{}

Dans le cadre de SR05, nous allons faire de \pie{} un réseau social proche de celui de \twitter{} avec cependant, quelques différences.
En effet, au lieu de fonctionner avec un système de serveurs centralisés au travers d'\internet, nous allons utiliser un système réparti composé d'un ensemble de véhicules équipés d'\airplug{} et de l'application \pie.
Un programme tel que \pie{} apporte un plus lors d'un trajet en voiture : non seulement nous pouvons suivre le flux de nos amis qui font route avec nous (en convoi), mais aussi de tout autre inconnu.
Ainsi, nous pouvons échanger un grand nombre d'informations avec les usagers nous entourant (dans la zone de diffusion).
Grâce au système d'abonnement, nous pouvons sélectionner les flux qui nous intéressent (collègues, informations routières, famille), tout en ignorant les autres (publicités, utilisateurs inintéressants\ldots).

