%mainfile: rapport.tex

\section{Terminologie}

\begin{description}
	\item[abonnement]

	Lors de l'utilisation de PIE, un utilisateur A voulant suivre le flux d'�v�nements d'un utilisateur B doit s'abonner � l'utilisateur B.
A est un abonn� (follower en anglais) de B. B ne suit pas pour autant les flux de A, PIE est donc un r�seau social asym�trique.
B ne pas peut refuser que A soit l'un de ses followers. Nous sommes dans un r�seau social publique, sans gestion de la confidentialit�, ou de liste de personnes autoris�es ou de personnes ind�sirables.

	\item[publication]

	La publication est l'action de mettre sur le r�seau un message de PIE. Ces messages ont une longueur maximale de 140 caract�res (tel Twitter), et ne peuvent contenir que du texte.  

	\item[voisins]

	Les voisins sont les sites (v�hicules) avec qui nous pouvons communiquer directement sans passer par un autre site. Ces noeuds sont forcements dans le rayon d'�mission direct des moyens de communication du noeud (wifi pour les voitures)


\end{description}

