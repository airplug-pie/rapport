%mainfile: rapport.tex

\section{Terminologie}

\begin{description}

\item[N\oe ud]
Dans le référentiel \airplug, un n\oe ud correspond simplement à une voiture équipée de la suite logicielle.

\item[Abonnement]
Lors de l'utilisation de \pie, un utilisateur A voulant suivre le flux d'évènements d'un utilisateur B doit s'abonner à l'utilisateur B.
A est un abonné de B. B ne suit pas pour autant les flux de A, PIE est donc un réseau social dit \emph{asymétrique}.
B ne pas peut refuser que A soit l'un de ses abonnés. Nous sommes dans un réseau social publique, sans gestion de la confidentialité, de liste de personnes autorisées ou de personnes indésirables.

\item[Publication]
La publication est l'action de mettre sur le réseau un message \pie. Ce dernier a une longueur maximale de 140 caractères (comme \twitter), et ne peut contenir que du texte.  

\item[Flux]
Le concept de flux correspond à l'ensemble des publications qu'un utilisateur reçoit des n\oe uds auxquels il est abonné. Il contient également les propres publications de l'utilisateur.

\item[Voisins]
Les voisins sont les n\oe uds avec lesquels un n\oe ud courant peut communiquer directement, sans passer par des intermédiaires. Ils sont forcément dans le rayon d'émission directe des moyens de communication du n\oe ud courant (le \wifi{} dans le cas des voitures).

\end{description}

