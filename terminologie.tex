%mainfile: rapport.tex

\section{Terminologie}

\begin{description}

\item[N\oe ud] TODO

\item[abonnement]
Lors de l'utilisation de PIE, un utilisateur A voulant suivre le flux d'évènements d'un utilisateur B doit s'abonner à l'utilisateur B.
A est un abonné (follower en anglais) de B. B ne suit pas pour autant les flux de A, PIE est donc un réseau social asymétrique.
B ne pas peut refuser que A soit l'un de ses followers. Nous sommes dans un réseau social publique, sans gestion de la confidentialité, ou de liste de personnes autorisées ou de personnes indésirables.

\item[publication]
La publication est l'action de mettre sur le réseau un message de PIE. Ces messages ont une longueur maximale de 140 caractères (tel Twitter), et ne peuvent contenir que du texte.  

\item[voisins]
Les voisins sont les sites (véhicules) avec qui nous pouvons communiquer directement sans passer par un autre site. Ces noeuds sont forcements dans le rayon d'émission direct des moyens de communication du noeud (wifi pour les voitures)

\end{description}

