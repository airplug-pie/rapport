\documentclass[a4paper,oneside,12pt]{article}

\usepackage[utf8]{inputenc}
\usepackage[T1]{fontenc}
\usepackage[french]{babel}
\usepackage{hyperref}
\usepackage{listings}
\usepackage{listingsutf8}
\usepackage{graphicx}


\newcommand{\airplug}{AIRPLUG}
\newcommand{\pie}{PIE}
\newcommand{\timestamp}{\textit{timestamp}}
\newcommand{\mdcinq}{MD5}
\newcommand{\hash}{\textit{hash}}

\newcommand{\chrisb}{Christophe Boudet}
\newcommand{\julien}{Julien Castaigne}
\newcommand{\chrisr}{Christophe Roquette}
\newcommand{\joe}{Jonathan Roudière}
\newcommand{\jeremy}{Jérémy Subtil}

\newcommand{\format}[1]{\paragraph{Format} \texttt{#1}}
\newcommand{\formatvar}[2]{\paragraph{Format} \texttt{#1}~$\Rightarrow$ \texttt{#2}}


\title{Projet SR05 -- Réalisation d'une application répartie sur la plateforme \airplug}
\author{\chrisb, \julien, \chrisr,\\\joe, \jeremy}

\makeatletter
\hypersetup{
	colorlinks=true,
	linkcolor=blue,
	citecolor=blue,
	urlcolor=blue,
	pdftitle={\@title},
	pdfauthor={\chrisb, \julien, \chrisr, \joe, \jeremy},
	pdfsubject={\pie}
}
\makeatother

\lstdefinelanguage{algorithme}
{keywords={si,alors,finsi,sinon,pour,faire,finfaire,et,ou,non,envoyer,recevoir},
otherkeywords={<-},
comment=[l]//,
morecomment=[l][\bfseries\normalsize]**}
\lstset{language=algorithme,
frame=single,
numbers=left,
tabsize=2,
stepnumber=2,
extendedchars=true,
numberstyle=\tiny,
basicstyle=\small,
extendedchars=true,
inputencoding=utf8/latin1}


\begin{document}

\maketitle
\tableofcontents
\clearpage

%mainfile: rapport.tex

\section{Présentation du sujet}

\paragraph{}

Le choix de notre projet SR05 est de reproduire un réseau social tel que \twitter, de manière répartie, par le biais du système \airplug.

\paragraph{}

\twitter{} est un réseau social permettant à ses membres de publier sur \internet{} de courts messages, de l'ordre de 140 caractères.
Ses utilisateurs peuvent s'abonner à d'autres membres du réseau social, dans le but de pouvoir lire les messages rédigés par ces derniers. L'utilisation qui est faite de la plateforme est variée : garder le contact avec ses amis, discuter, publier une revue de presse personnelle ou encore couvrir des évènements publics en temps réel.

\twitter{} est par ailleurs utilisé par de grandes entreprises, des politiciens, dans le but de pouvoir être lus par plusieurs millions de personnes à travers le monde, mais aussi pour avoir un retour sur leurs actions.
Il n'est pas rare de voir un service après-vente tel que celui de \dell, vous contacter si vous vous plaignez d'un problème avec un de leurs produits sur \twitter.

\paragraph{}

Dans le cadre de SR05, nous allons faire de \pie{} un réseau social proche de celui de \twitter{} avec cependant, quelques différences.
En effet, au lieu de fonctionner avec un système de serveurs centralisés au travers d'\internet, nous allons utiliser un système réparti composé d'un ensemble de véhicules équipés d'\airplug{} et de l'application \pie.
Un programme tel que \pie{} apporte un plus lors d'un trajet en voiture : non seulement nous pouvons suivre le flux de nos amis qui font route avec nous (en convoi), mais aussi de tout autre inconnu.
Ainsi, nous pouvons échanger un grand nombre d'informations avec les usagers nous entourant (dans la zone de diffusion).
Grâce au système d'abonnement, nous pouvons sélectionner les flux qui nous intéressent (collègues, informations routières, famille), tout en ignorant les autres (publicités, utilisateurs inintéressants\ldots).


%mainfile: rapport.tex

\section{Terminologie}

TODO julien/chrisB: définir les termes qu'on utilise pour décrire notre appli. Si c'est trop redondant avec la présentation, un zappe cette partie.

\begin{description}
	\item[abonnement]
	\item[publication]
	\item[voisins]
\end{description}


%mainfile: rapport.tex

\section{Algorithme réparti}

\lstinputlisting[caption={Algorithme réparti}]{algo.txt}


%mainfile: rapport.tex

\section{Protocole}

TODO Jérémy


% JOE : cela me semble plus logique de mettre l'avancement à la fin et de présenté (comme précédement)
% les structures de données avant l'IHM, si quelque veut changer cela ... (juste alors faire gaffe à la mise en page)
%mainfile: rapport.tex


\section{Structures de données}
\label{section:storage}

\subsection{Le modèle -- vue d'ensemble}

\paragraph{}

Le modèle des structures de données utilisées par PIE est basé autour de la notion de
flux, à ce flux est associé un état. Ces flux sont caractérisés par un couple d'informations
garantissant leur unicité. Il s'agit de la provenance du flux -- le champ \texttt{FROM} du protocole
(\cfsection{mesg}) -- et de l'identité de l'utilisateur associé a ce flux -- le champs \texttt{NICK}
du protocole (\cfsection{mesg}).


\paragraph{}

En plus de celles précitées, un flux est un objet constitué d'un ensemble de caractéristiques.
Ces caractéristiques sont d'une part les informations liées à la notion d'identité de utilisateur
associé au flux, de l'autre celles liées -- au même titre que l'état -- à sa gestion.


\paragraph{}

Les informations caractérisant un utilisateur associé à un flux ne sont pas -- par défaut --
transmises par celui-ci (à l'exception du couple \texttt{NICK}/\texttt{FROM}). Lorsque 
l'utilisateur local de PIE en fait la demande, une requête unidirectionnelle (de type \texttt{getinfo})
peut alors être adressée au flux en question; si une réponse de sa part est reçu alors
les informations le concernant sont mises à jour.


\paragraph{}

Du fait de leur unicité, les flux constituent un ensemble, une collection d'objets uniques (\cfsection{vin}).
L'état définit une relation entre un flux et l'utilisateur local de l'application. Un flux
peut être :

\begin{description}
	\item[\texttt{available}] l'utilisateur local a accès à ce flux,
    \item[\texttt{forwarded}] l'utilisateur local transmet le flux à des paires,
    \item[\texttt{subscribed}] l'utilisateur local est abonné au flux.
\end{description}


\paragraph{}

Ces états possibles se superposent, car un flux peut avoir les trois états simultanément.
L'état d'un flux est une caractéristique offrant les relations ensemblistes suivantes :

\begin{itemize}
	\item quelque soit un flux x alors il appartient à \texttt{available},
	\item \texttt{subscribed} est un sous ensemble d'\texttt{available},
	\item \texttt{forwarded} est un sous ensemble d'\texttt{available},
	\item \texttt{subscribed} U \texttt{forwarded} est inclus ou égal à \texttt{available},
	\item l'intersection de \texttt{subscribed} et \texttt{forwarded} est éventuellement non vide.
\end{itemize}


\paragraph{}

L'état d'un flux est une caractéristique de gestion, cette caractéristique est modélisée
par l'utilisation de trois listes :

\begin{itemize}
	\item la liste des flux \texttt{available}, soit l'ensemble des flux,
	\item la liste des flux \texttt{forwarded}, le sous-ensemble des flux transmis,
	\item la liste des flux \texttt{subscribed}, le sous-ensemble des flux auquel l'utilisateur local est abonné.
\end{itemize}


\paragraph{}

Un gestion unifiée de ces structures de données est offerte par un sous-système de PIE,
l'interface de stockage. Cette interface permet la manipulation des flux, des informations 
qui leur sont associées et de leurs états, c'est l'interface de plus haut niveau pour la
manipulation des flux. L'interface de stockage est une interface de gestion, automatisant
les différentes taches inhérentes à cette gestion, elle met à profit les relations ensemblistes 
exposées à travers l'utilisation de listes.


\paragraph{}

Le schéma suivant présente les structures de données précitées et les relations (logiques)
entre chacune :

\begin{center}
    \includegraphics[width=0.8\textwidth]{img/struct.png}
\end{center}


\subsection{Implémentation}

\paragraph{}

Le modèle présenté dans le précèdent paragraphe a conduit à l'implémentation
présentée maintenant. Chacun des objets considérés dans le modèle dispose d'une
interface qui lui est propre. Il y a donc quatre interfaces de gestion différentes :

\begin{itemize}
	\item \texttt{user} : interface de gestion des utilisateurs associés aux flux,
	\item \texttt{stream} : interface de gestion des flux,
	\item \texttt{list}:  interface de gestion des listes de flux,
	\item \texttt{storage} : l'interface de stockage, plus haut niveau d'abstraction.
\end{itemize}


\paragraph{}

Chacune des interfaces précitées offre une couche d'abstraction permettant la manipulation
des objets qu'elle modélise. Elles sont définies par deux niveaux : il y a d'une part, des
fonctions de manipulation globales des objets fournis par l'interface (telle que seraient
des fonctions statiques d'une classe en Java) et de l'autre ,des fonctions de manipulation
des objets par eux mêmes (le pendant des méthodes en Java). Les fonctions et méthodes des
différentes interfaces permettent par exemple la recherche d'objets en fonction de la valeur
de leur attributs autant que leur mise à jour ou encore leur destruction.


\subsubsection{Interface User}

\paragraph{}

Un utilisateur (\texttt{user}) est représenté par un objet disposant
d'un certain nombre de caractéristiques, celles-ci sont :

\begin{itemize}
	\item \texttt{id} : information de gestion initialisée à la création,
	\item \texttt{nickname}	: correspond au champ \texttt{NICK} définit dans le protocole,
	\item \texttt{email} : l'adresse e-mail de l'utilisateur (optionnelle),
	\item \texttt{fullname} : son nom (optionnelle),
	\item \texttt{firstname} : son prénom (optionnelle),
	\item \texttt{phone\_nb} : son numéro de téléphone (optionnelle),
	\item \texttt{age} : son âge (optionnelle),
	\item \texttt{sex} : son sexe (optionnelle),
	\item \texttt{dest} : sa destination actuelle (optionnelle),
	\item \texttt{desc} : se description de l'utilisateur (optionnelle).
\end{itemize}


\paragraph{}

Les caractéristiques notées optionnelles sont celles n'ayant pas d'impact
sur le fonctionnement de PIE en interne, elles n'ont pour objet que d'offrir
un profil plus détaillé (informations obtenues par une requête \texttt{getinfo}).
 
 
\subsubsection{Interface Stream}

\paragraph{}

Ces objets utilisateur, sont implémentés en utilisant le package Itcl qui offre
une couche abstraction de type objet au langage TCL, ils sont contenus
(associés) dans un objet de type flux (\texttt{stream}). L'objet \texttt{stream}
est défini par les champs :

\begin{itemize}
	\item \texttt{stream\_id} : information de gestion initialisée à la création,
	\item \texttt{car\_id} : information correspondant au champ \texttt{FROM} du protocole,
	\item \texttt{time\_available} : l'heure locale depuis laquelle le flux est disponible,
	\item \texttt{time\_lastmsg} : l'heure locale du dernier message reçu de ce flux,
	\item \texttt{time\_lasthello} : l'heure locale du dernier message de type \heartbeat,
	\item \texttt{nb\_mesg} : le nombre de messages envoyés par ce flux,
	\item \texttt{user} : l'objet utilisateur présenté dans le précédent paragraphe,
	\item un certain nombre d'informations de gestion pour le protocole.
\end{itemize}


\subsubsection{Interface List}

Les listes bien que nativement présentent dans le langage TCL bénéficient également
d'une interface particulière et écrite par nos soins. Cette interface permet une gestion
cohérente au modèle ensembliste -- unicité des éléments dans chaque liste.


\subsubsection{Interface Storage}

\paragraph{}

Enfin l'interface de stockage est celle permettant la gestion de l'ensemble, c'est 
celle présentant le plus haut niveau d'abstraction, c'est donc à travers cette interface
que sont manipulés tous les objets du modèle. Le stockage maintient quatre listes :

\begin{itemize}
	\item la liste des flux \texttt{available}, soit l'ensemble des flux,
	\item la liste des flux \texttt{forwarded}, sous-ensemble des flux transmis,
	\item la liste des flux \texttt{subscribed}, sous-ensemble des flux souscrits localement,
	\item la liste des flux \texttt{forgotten}, les flux devenus indisponible.
\end{itemize}


\paragraph{}

Lorsque un nouveau flux est crée, cela correspond à la réception d'informations de sa part, l'objet
associé à ce flux est alors initialisé. Il est crée à travers l'interface de stockage des flux et
bénéficie de l'état disponible (\texttt{available}). Si l'utilisateur décide de s'y abonner, alors
le flux bénéficie également de cet état et à ce titre, est mis dans la liste associée (\texttt{subscribed}).
Enfin un flux pour lequel PIE est informé qu'il intéresse un pair, sera mis dans la liste des flux
transmis (\texttt{forwarded}).


\paragraph{}

Lorsqu'un flux devient indisponible, plutôt que de détruire l'objet associé, celui-ci est placé dans la
liste \texttt{forgotten}, il tombe dans les limbes. Il s'agit d'une optimisation et d'un écart au modèle
présenté : ne pas détruire ce flux permet de le rendre disponible de nouveau à l'utilisateur local et
cela sans sur-coup au niveau des ressources locales. De plus, si d'aventure, des informations avaient
au préalable été obtenues sur l'utilisateur associé au dit flux (via une requête de type \texttt{getinfo}) alors
il n'est pas nécessaire de les redemander ensuite, ce qui évite de saturer le média de communication.


\paragraph{}

L'évolution des états d'un flux peut être représenté par le schéma suivant :

\begin{center}
    \includegraphics[width=1\textwidth]{img/state.png}
\end{center}

Les listes ne contiennent en fait que des références vers les objets. C'est à l'interface de
stockage de garantir les propriétés précédemment exposées.


%mainfile: rapport.tex

\section{Interface graphique (IHM)}
\label{section:ihm}

\subsection{Présentation}

L'interface graphique permet l'utilisation de PIE par l'utilisateur local. Elle a pour vocation
de présenter une vue la plus complète possible de l'état de l'application à l'utilisateur. Ceci
est réalisé à travers neuf onglets et une fenêtre. Un menu permet de changer l'état de l'application
et les vues (onglets/fenêtre) proposées.


\paragraph{PIE's interface :}
l'onglet principal. Il offre une vue à l'utilisateur des flux disponibles,
des flux auxquels l'utilisateur est abonné et une zone de texte afin que ce dernier puisse écrire et
envoyer des messages (voir capture ci-dessous). Cet onglet est tout le temps ouvert. Il est possible
de s'abonner ou se désabonner d'un flux en le faisant glisser de la zone référençant les flux disponibles
vers celles référençant les flux auxquels l'utilisateur est abonné et vice-versa (drag and drop).

\begin{center}
    \includegraphics[scale=0.5]{img/pie-main.png}
\end{center}

\clearpage


\paragraph{Current Profile :}
onglet permettant la consultation et la modification du profil
courant de l'utilisateur local. Si le profil est modifié, il est alors enregistré.

\begin{center}
    \includegraphics[scale=0.5]{img/profile.png}
\end{center}


\paragraph{Global settings :}
onglet permettant la consultation et la modification des réglages 
d'ordre général de PIE (les onglets ouverts au démarage de l'application ou le mode par défaut).


\paragraph{Your messages :}
onglet qui permet de consulter les précédents messages envoyés par l'utilisateur local.

\clearpage


\paragraph{Forwarded streams :}
onglet qui permet de consulter les flux qui sont transmis à des pairs, de mettre à jour 
les informations affichées et éventuellement d'arrêter la transmission de ce flux.

\begin{center}
    \includegraphics[scale=0.5]{img/forwarded.png}
\end{center}

\clearpage


\paragraph{Subscribed streams :}
fenêtre permettant de voir les flux auxquels l'utilisateur local est abonné, chaque flux est affiché dans un onglet.
Les informations concernant le flux sont affichées ainsi que les messages qu'il a émis.

\begin{center}
    \includegraphics[scale=0.5]{img/subscribed.png}
\end{center}


\paragraph{Pie traces :}
onglet qui affiche toutes les traces (debug) de l'application et de ses sous-systèmes.


\paragraph{Network traces :} onglet qui permet de consulter l'ensemble des messages émis ou reçus depuis le réseau dans leur forme brute.


\paragraph{Input/Output traces :} deux onglets qui permettent de consulter l'ensemble des messages reçus/émis depuis le réseau dans leur forme brute.


\section{Comptes utilisateurs et configuration}

\paragraph{}

PIE utilise deux fichiers de configuration, le premier est celui définissant le profil de l'utilisateur
alors que le second permet de configurer l'application elle-même (onglet ouvert, mode, \ldots).


\paragraph{}

Les fichiers de configuration sont conservés dans le répertoire \texttt{\$HOME/.pie} (par défaut), le fichier
de configuration globale doit s'appeler \texttt{global.conf} et les fichiers de profil utilisateur doivent
s'appeler \texttt{NICKNAME.conf}.


\paragraph{}

À son lancement l'interface graphique vérifie à travers le sous-système de gestion des configurations
que le répertoire de configuration et les fichiers précités sont présents. Si ce n'est pas le cas, alors le
répertoire de configuration et un fichier de configuration globale générique sont crées. L'utilisateur, lui,
est invité à définir un profil via l'IHM.


\paragraph{}

Si plusieurs profils utilisateur sont disponibles alors l'utilisateur est invité à en choisir un. Une fois ces
vérifications faites et un profil déterminé l'application PIE est prête à être utilisée.


%mainfile: rapport.tex

\section{Identification unique des véhicules}
\label{section:vin}

TODO Julien
% ne pas hésiter à renommer cette partie


%mainfile: rapport.tex

\section{Bilan de l'avancement}

\paragraph{}

Les fonctionnalités actuellement offertes par notre application \pie{} sont les suivantes :

\begin{enumerate}
	\item l'utilisateur peut poster une publication \pie{} à travers le réseau ;
	\item il peut s'abonner aux flux de ses voisins ;
	\item il reçoit les publications des personnes auxquelles il est abonné, dans la mesure où les publications arrivent jusqu'à lui ;
	\item un prototype d'interface graphique complet est fourni (\cfsection{ihm}) ;
	\item plusieurs utilisateurs peuvent être gérés simultanément à travers plusieurs instances de \pie ;
	\item une configuration des préférences l'utilisateur est sauvée dans son répertoire personnel\footnote{\texttt{\$HOME/.pie/}}.
\end{enumerate}


\paragraph{}

D'un point de vue plus technique, les aspects suivants sont implémentés :

\begin{enumerate}
	\item le messages \pie{} et le \heartbeat{} sont gérés ;
	\item une librairie permettant de stocker les différents flux utiles au bon fonctionnement de \pie{} a été développée (\cfsection{storage}) ;
	\item un prototype de gestion d'identification unique de véhicule reposant sur le \vin{} est proposé (\cfsection{vin}).
\end{enumerate}


\paragraph{}

Toutefois, notre partie réalisation n'est pas encore terminée à la date de rédaction de ce rapport.
Nous n'avons donc pas pu déployer l'application de façon satisfaisante, de manière à effectuer de vrais tests pour valider ou critiquer les choix de notre algorithme réparti.

Nous avons également réfléchi à des extensions pour notre projet (\cfsection{evolutions}), qui seront éventuellement présentées lors de notre soutenance si le temps nous le permet.


%mainfile: rapport.tex

\section{Évolutions possibles}
\label{section:evolutions}

Dans cette version de \pie, nous nous sommes concentrés sur l'implémentation de notre algorithme et donc sur la spécificité du routage.
Afin de permettre une évolution de l'application, nous avons tenté de respecter au maximum la philosophie \airplug{} dans le nommage des variables et procédures.
D'autre part, chaque fonction est documentée et chaque structure a été pensée modulable afin de ne pas limiter les possibilités de \pie{} à sa version actuelle.

\subsection{Gestion des utilisateurs}
\airplug{} donne la possibilité aux utilisateurs d'une voiture de communiquer grâce à des applications via la borne \wifi. Mais que se passe-t-il si plusieurs personnes dans la voiture décident d'utiliser \airplug{} ?
Dans un tel cas, l'identité de la voiture ne suffit plus, mais il faut identifier chaque personne de la voiture. Pour ce faire, nous avons mis en place la structure d'utilisateurs évoquée précédemment. Dans le contexte de \pie, nous pouvons envisager rapidement de permettre à un utilisateur de récupérer les informations des propriétaires des flux qu'il suit. Ceci pourra être réalisé simplement par l'utilisation de deux nouveaux types de messages : un type pour effectuer la demande d'information, et un autre pour la réponse. Dans le cas d'une demande, nous pouvons imaginer qu'étant donné que chaque n\oe ud peut posséder l'information, il ne sera pas nécessaire que la réponse provienne du propriétaire du flux, mais de toute personne possédant ces données.

\subsection{Connaissance des abonnés}
Dans la version actuelle de \pie, il n'est pas possible pour un utilisateur de connaître l'identité des personnes intéressées par son flux. 
Cependant, cela pourrait être envisagé en ajoutant un type de message spécial demandant à chaque abonné de décliner son identité au propriétaire du flux. 
Il est par ailleurs possible de réaliser une première liste grâce au partage des demandes entre les n\oe uds. Ainsi, en ajoutant une information sur la personne qui demande, nous pourrions croiser nos données en regardant dans notre liste \texttt{forwarded} les n\oe uds qui désirent notre flux. Cette liste ne serait pas complète étant donné que nous n'avons pas forcément reçu la demande de tous les n\oe uds, même éloignés.

\subsection{Des applications tierces}
En ajoutant la gestion des identités, nous pouvons imaginer des applications tierces, utilisant \pie{} comme couche pour l'envoi de messages. Ces applications pourraient entre autres utiliser les destinations des utilisateurs pour proposer divers services comme du guidage, des points de rencontre dans le cas de longs trajets.


%mainfile: rapport.tex
\section{Conclusion}
Ce projet de SR05 nous a permis d'imaginer une partie des problèmes soulevés par les algorithmes répartis. Il était jusqu'à maintenant rare, pour nous, d'imaginer une application ne s'appuyant sur aucune plateforme centralisée. Ce changement de façon de penser nous a poussés à améliorer nos algorithmes et à réfléchir sur des problèmes jusqu'alors simples dans système centralisé comme le partage d'information, l'acheminement des messages, la topologie du réseau.



\end{document}

