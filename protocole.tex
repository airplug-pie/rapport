%mainfile: rapport.tex

\section{Protocole}

La définition du protocole de l'application \pie{} passe par la description des identifiants utilisés dans le système, ainsi que de celle du format des messages qui s'échangent dans le réseau.


\subsection{Identifiants}

Pour des raisons pratiques, un certain nombre d'objets de notre système doivent être identifiés.


\paragraph{Le n\oe ud}

Son identifiant est celui qui a été fourni à \airplug{} lors de son lancement.

\format{\fvnodeid}


\paragraph{La publication}

Son identifiant est défini par un \hash{}, calculé à partir de la date de sa rédaction et de l'identifiant du n\oe ud auteur.

\format{\fvmsgid}


\subsection{Formats de messages}

Les messages de l'application \pie{} sont composés :
\begin{enumerate}
	\item d'un entête ;
	\item d'un \payload, dont le format varie selon le type de message.
\end{enumerate}


\subsubsection{Entête}

Un entête commun a été défini pour tous les messages envoyés par l'application \pie.
Ses champs sont les suivants :

\begin{description}
	\item[\fkfrom] désigne l'émetteur d'origine du message. Ce champ n'est pas modifié en cas de relais du message par un n\oe ud voisin.
	\item[\fknick] désigne le pseudo de l'émetteur d'origine du message. Ce champ n'est pas non plus modifié en cas de relais.
	\item[\fkttl] correspond au nombre de sauts possibles parmi les n\oe uds n'ayant pas de voisin intéressé par le message.
	\item[\fktts] correspond au nombre de sauts possibles parmi les n\oe uds ayant un voisin intéressé par le message.
	\item[\fktype] indique le type du message, impliquant un certain format de \payload. Ce champ peut prendre les valeurs suivantes :
		\begin{itemize}
			\item \texttt{\msgheartbeat{} = 0}
			\item \texttt{\msgpie{} = 1}
		\end{itemize}
\end{description}

\format{\ffrom \apgdelim \fnick \apgdelim \fttl \apgdelim \ftts \apgdelim \\ \ftype \apgdelim \fvpayload}


\subsubsection{\Payload : message \msgheartbeat}

Le \heartbeat{} correspond à une annonce des offres et des demandes du n\oe ud lui-même ou de ses voisins, directs ou non.

Le \payload{} du message contient donc à la fois une liste d'offres et une liste de demandes.
Une entrée de ces listes comprend un identifiant de n\oe ud et une pondération associée.

Du fait de la taille limitée des messages \airplug, toutes les offres et demandes disponibles ne peuvent pas être annoncées via le message \msgheartbeat.
Ainsi, les entrées doivent être sélectionnées :

\begin{enumerate}
	\item les entrées correspondant aux n\oe ud sont choisies en priorité ;
	\item les entrées les plus importantes sont sélectionnées ;
	\item enfin, s'il reste de la place, on choisit aléatoirement des entrées pour compléter.
\end{enumerate}

Par ailleurs, les messages \msgheartbeat{} sont utilisés par un n\oe ud pour connaitre ses voisins.
En effet, il suffit de récolter le contenu des champs \fkfrom{} et \fknick{} de leurs entêtes.

\format{\foffers \apgdelim \fdemands}

\formatvar{\fvoffers}{\\\fvnodeid-\fvnick-\fvdistance,\fvnodeid-\fvnick-\fvdistance,\ldots}

\formatvar{\fvdemands}{\\\fvnodeid-\fvnick-\fvdistance,\fvnodeid-\fvnick-\fvdistance,\ldots}


\subsubsection{\Payload : message \msgpie}

Le message \msgpie{} correspond à l'envoi d'une publication d'un utilisateur.
Il contient alors les champs suivants :

\begin{description}
	\item[\fkmsgid] identifie la publication de façon unique. Il permet entre autres de savoir si un même message a été reçu plusieurs fois.
	\item[\fkmsgdate] correspond à la date de la rédaction de la publication.
	\item[\fkmsgcontent] correspond au contenu de la publication.
\end{description}

\format{\fmsgid \apgdelim \fmsgdate \apgdelim \fmsgcontent}

